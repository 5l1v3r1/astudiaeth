\documentclass[12pt,a4paper]{report}
\usepackage[francais]{babel}
\usepackage[pdftex]{graphicx}
\usepackage [utf8x]{inputenc}
\usepackage{amsmath}
\usepackage{amsfonts}
\usepackage{amssymb}
\usepackage{algorithm}
\usepackage{algorithmic}
\usepackage{subfigure}

\floatname{algorithm}{Algorithme}

\let\mylistof\listof
\renewcommand\listof[2]{\mylistof{algorithm}{Liste des algorithmes}}

\makeatletter
\providecommand*{\toclevel@algorithm}{0}
\makeatother


\title{Théorie de la complexité}

\begin{document}

\maketitle
\tableofcontents
\newpage 

\chapter{Introduction}
\section{Approche semi-formelle}
\begin{itemize}
\item Factorisation des entiers :
\begin{itemize}
\item instances du problème (données)
\item question 
\end{itemize}
$n \longrightarrow  \bigcirc \longrightarrow p_1, k_1, \ldots, p_m, k_m $ (o $p_1^{k_1}\ldots p_m^{k_m} = n$)
\item Calcul dans des graphes
\begin{figure}[h]
	\centering
  \scalebox{0.5}{\input{graphe1.pdf_t}}
\end{figure}
$$ G = (S,A) $$
$S$ fini, $|S|=n$, $A \subset \{$ paires de sommets $\}$ \\

Questions :
\begin{enumerate}
\item Existe-t-il un "triangle" ?
\item Quel est le plus court chemin de a vers b ?
\item Existe-t-il un circuit hamiltonien ? (tous les sommets une seule fois)
\end{enumerate}
\item Calculer un PGCD
\item Calculer un exposant RSA
\end{itemize}
L'existence d'un algorithme efficace qui donne la réponse revient à dire que le problème est "facile".\\
\paragraph{"Définition" : \\}
Un problème de décision est un problème dont la réponse est oui ou non.\\
But: Faire la différence entre :
\begin{itemize}
\item Problèmes "faciles" : algorithme rapide
\item Problèmes "difficiles" : pas d'algorithme rapide
\end{itemize}

entrée de taille n $\longrightarrow \bigcirc \longrightarrow $ sortie après un temps $ \leqslant f(n) $ \\
Un problème facile a un temps $\leqslant n^k$ avec $ k$ fixé (temps polynomial) et un problème difficile n'a pas de k qui vérifie cette condition.\\

\paragraph{Quelle est la taille de l'entrée ?\\}
Elle est exprimée en nombre de bits (symboles). \\
Un graphe peut être représenté par sa matrice d'adjacence de taille $n^2$ mais si le graphe a peu d'arêtes on peut le représenter par des listes de voisins qui ont une taille de $n^2 \log n$ mais s'il y a peu d'arêtes on a en réalité $n\log n$.\\
On peut toujours se ramener à un problème de décision.\\

Exemple : la factorisation des entiers. On a en entrée $N, M$ entier $\leqslant N$ de taille $n=\log N$.\\
Question : Existe-t-il un diviseur $d$ de $N$ tel que $1 < d \leqslant M$ ?
\begin{figure}[h]
	\centering
  \scalebox{0.8}{\input{facto.pdf_t}}
\end{figure}

On utilise la dichotomie pour résoudre le problème initial en prenant $M = \frac{N}{2} $. On a $\log n$ questions à l'oracle pour $d$.

\section{Passage du semi-formel au formel}
Un problème de décision est un langage. Une instance est un élément de $\Sigma ^*$ où $\Sigma ^*$ est un alphabet fini (par exemple $ \Sigma = \{0,1\} $ ). Le langage $ L \subset \Sigma $ est l'ensemble des instances "positives" (dont la réponse est oui). On doit reconnaitre L avec une machine. 
\paragraph{Qu'est-ce qu'une machine ?\\}
Une machine simple est un automate. Par exemple, l'automate qui suit 
\begin{figure}[h]
	\centering
  \scalebox{0.5}{\input{automate.pdf_t}}
\end{figure}
 accepte les mots se finissant par 1.\\
Un langage reconnaissable par un automate est un langage rationnel (régulier).\\

Machine de Turing :\\

\begin{tabular}{c|c|c|c|c|c|c|c}\hline
-3 & -2 & -1 & 0 & 1 & 2 & 3 & 4 \\\hline 
b &b& b& b& $x_1$ & $x_2$ & $x_3$ & b \\\hline
\end{tabular}\\

alphabets : $\Sigma$  et $\Gamma \supset \Sigma \cup \{b\} $\\

Opération élémentaire (transition) :
\begin{itemize}
\item La machine est dans l'état $q_i$ et lit le symbole $s \in \Gamma$.
\item Elle écrit $t$ (au lieu de $s$), $ t\in \Gamma$.
\item Elle se déplace de $\{-1,0,1\}$.
\item Elle change d'état $q_i \rightarrow q_j$, $Q=\{q_0,\ldots,q_n,q_A,q_R\}$.
\end{itemize}
Fonction de transition $\delta$ : $Q\times \Gamma \longrightarrow Q\times \Gamma \times \{-1,0,1\}$\\
Calcul : entre : $x_1,x_2,\ldots, x_m$, curseur en position 1, état $q_0$. La suite de transition s'arrête si on a $q_A$ ou $q_R$.\\
Temps de calcul : $t_m(x) = $ nombre de transitions jusqu'à $q_A$ ou $q_R$ (on peut avoir $\infty$). Le langage reconnu par M est : 
$L_M =\{x \in \Sigma^*, M $ accepte $ x\}$.\\
Tableau :\\

\begin{tabular}{|c|c|c|c|}\hline
 & b & 0 & 1 \\\hline
$q_0$ & b,R,0 & 0,$q_0$,$+1$ & 1,$q_1$,$+1$ \\\hline
$q_1$ & b,A,0 & 0,$q_0$,$+1$ & 1,$q_1$,$+1$ \\\hline
\end{tabular}\\
En résumé :
\begin{itemize}
\item Algorithme = machine de Turing 
\item Problème = Langage ($\subset \Sigma^*$)
\end{itemize}
On a $t_M$ qui est la complexité de la machine M (en temps)

\paragraph{Définition :} machine de Turing universelle (ordinateur)\\
La machine $U$ prend comme entrée $<M>$ qui est la description de la machine M et $x$ et U retourne ce que retourne M sur l'entrée $x$.
\paragraph{Théorème :\\}
Il existe des machines de Turing universelles.\\
On a $t_U(<M>, x) \leqslant c |x| t_M(x)$ avec $|x|$ longueur de $x$.

\chapter{Complexité en temps}
\section{Problèmes faciles ?}
\paragraph{Définition :\\}
La classe P est l'ensemble des langages polynomiaux.
\paragraph{Définition :\\}
L est un langage polynomial si $\exists$ M et un polynôme $P(X) \in \mathbb{Z}[X]$ tels que $ \forall x \in \Sigma^*,\ t_M(x) \leqslant P(|x|)$ et M reconnait L.

\section{Problèmes difficiles ?}
\paragraph{Exemples :}
\begin{itemize}
\item Circuit hamiltonien (CH) :
\begin{itemize}
\item[I:] graphe
\item[Q:] existe-t-il un circuit hamiltonien dans ce graphe ?
\end{itemize}
\item Satisfaisabilité  (SAT) : 
\begin{itemize}
\item[I:] formule booléenne : $f=(x_1 \cup x_2 \cup \bar{x_4} ) \cap (x_2 \cup \bar{x_5})$
\item[Q:] $\exists ? (a_1,\ldots,a_n) \in \{0,1\}^n,\ f(a_1,\ldots, a_n)= 1$ (vrai)
\end{itemize}
\item FBQ :
\begin{itemize}
\item[I:] $\forall x_1, \exists x_2, \forall x_3,x_4,\ f(x_1,\ldots,x_n)= 1$
\item[Q:] Vrai ou faux ?
\end{itemize}
\item Clique
\begin{itemize}
\item[I:] graphe G, $k \in \mathbb{N}$
\item[Q:] existe-t-il une clique (sous graphe complet) de taille $k$  ?
\end{itemize}
\item 10ème problème de Hilbert :
\begin{itemize}
\item[I:] un polynôme $P(X_1,\ldots,X_n) \in \mathbb{Z}[X_1\ldots X_n]$
\item[Q:] $ \exists (a_1,\ldots,a_n) \in \mathbb{Z}^n,\ P(a_1,\ldots,a_n) = 0$ ?
\end{itemize}
\item Voyageur de commerce (VC) :
\begin{itemize}
\item[I:] graphe G, $p_1,\ldots p_m \in \mathbb{N}, k \in \mathbb{N}$ avec $p_i$ poids de $a_i$
\item[Q:] Existe-t-il un parcours de tous les sommets du graphe où la somme des longueurs des arêtes parcourues est $<k$ ?
\end{itemize}
\item k-coloration :
\begin{itemize}
\item[I:] graphe, $k\in \mathbb{N}$
\item[Q:] Peut-on colorier les sommets avec $k$-couleurs ?
\end{itemize}
\end{itemize}
\paragraph{Définition :\\}
Un langage L est dit certifiable en temps polynomial si $\exists M$ et $P(X)$ tel que pour tout $x \in L$, $\exists y \in \Sigma^*$ tel que sur l'entrée $(x,y)$, M aboutit à l'état $q_A$ en temps :
$$ t_M(x,y) \leqslant P(|x|) $$
D'où, la classe NP est l'ensemble des langages certifiables en temps polynomial.\\
Conjecture : $P \neq NP$.
\paragraph{Réduction polynomiale\\}
$\varphi : \Sigma^* \longrightarrow \Sigma^*$ est une réduction polynomiale de L vers L' si :
\begin{itemize}
\item $\varphi$ est calculable en temps polynomial
\item $l\in L \Longleftrightarrow \varphi(l) \in L'$
\end{itemize}
\paragraph{Définition :\\}
$L \propto L'$ si $\exists \varphi$ réduction de L vers L'.\\
Exemple : CH $ \propto$ VC, $\varphi : G \rightarrow (G,p_1=\ldots =p_n=1,k=n)$
\paragraph{Définition : }Complétude \\
$\Lambda$ est NP-complet si :
\begin{itemize}
\item $\Lambda \in NP$
\item $\forall L \in NP,\ L \propto \Lambda$
\end{itemize}
\paragraph{Théorème :\\}
Il existe un problème NP-complet.
\subparagraph{Exemple d'un tel problème :} Arrêt en temps limité
\begin{itemize}
\item[I:] $<M>,\ x \in \Sigma^*,\ 1^t = 11\ldots1\ (\mbox{t fois})$
\item[Q:] Est qu'il existe $y \in \Sigma^*$ tel que $M$ s'arrête sur $q_A$ sur l'entrée $(x,y)$ en temps $\leqslant t$ ?
\end{itemize}
\subparagraph{Théorème :\\}
L'arrêt en temps limité est NP-complet.
\subparagraph{Preuve :\\}
\begin{itemize}
\item Problème dans NP ?\\
Le certificat est $y$. Soit U une machine de Turing universelle, U exécute le déroulement de $M$ sur l'entrée $(x,y)$ et $1^t$.
\item $\forall L \in NP,\ L \propto \Lambda $ ?\\
$L$ vient avec $M_L$, $P$ un polynôme, tels que $x \in L \Longleftrightarrow \exists L,\ M_L $ accepte $(x,y)$ en temps $\leqslant P(|x|)$
$$ x \longmapsto <M_l>,\ x,\ 1^{P(|x|)}$$
$x \in L \Longleftrightarrow \exists y$ tel que $M_L$ s'arrête sur $(x,y)$ en temps $\leqslant P(|x|)$\\
$ x \in L \Longleftrightarrow f(x) \in \Lambda $
\end{itemize}
\paragraph{Théorème de Cook\\}
SAT est NP-complet.
\paragraph{Exemple de réduction polynomiale \\}
Le problème CLIQUE :
\begin{itemize}
\item[I:]G graphe, $k \in \mathbb{N}$
\item[Q:]existe-il une clique de G (sous graphe complet) de taille $k$ ?
\end{itemize}
\begin{figure}[h]
	\centering
  \scalebox{0.8}{\input{clique4.pdf_t}}
\end{figure}

\subparagraph{Proposition :\\}
CLIQUE est NP-complet.
\subparagraph{Preuve :\\}
$$ SAT \propto CLIQUE $$
\begin{figure}[h]
	\centering
  \scalebox{0.8}{\input{schema_sat_clique.pdf_t}}
\end{figure}


\subparagraph{Proposition :\\}
Si $\Lambda$ est NP-complet et si $\Lambda \propto L$ et $L\in NP$ alors $L$ est NP-complet. (car si $L \propto L'$ et $L' \propto L''$ alors $L \propto L''$)\\

Exemple : $f=(x_1 \cup x_2 \cup x_3) \cap (x_1 \cup \bar{x_2} \cup x_4) \cap (\bar{x_1} \cup x_3 \cup \bar{x_4})$ donne un graphe G avec $ \#$sommets $=$ longueur de f $=\#$litéraux.

\begin{figure}[h]
	\centering
  \scalebox{0.7}{\input{exemple_clique.pdf_t}}
\end{figure}

Il faut des arêtes partout sauf :
\begin{itemize}
\item entre sommets issus d'une même clause
\item entre 2 sommets issus de $x_i$ et $\bar{x_i}$
\end{itemize}
$k=\#$clauses (ici $k=3$)\\
f satisfaisable $\Leftrightarrow $ G admet une $k$-clique.\\
$\Rightarrow$ : Si f est satisfaisable, il existe dans chaque clause, un litéral de valeur 1. Cela donne une clique dans G.\\
$ \Leftarrow $ : Si on a une $k$-clique, il est possible de mettre un terme dans chaque clause à 1.
\section{Classe NP et fonctions à sens unique}
\paragraph{Définition :} Fonction à sens unique\\
$f : \Sigma^* \longrightarrow \Sigma^*$
\begin{itemize}
\item $Dom(f)$ est infini
\item f est calculable en temps polynomial
\item $\not\exists  M$ machine, polynome $P$ tel que $\forall y \in Im(f)$, $M$ calcule $x$ tel que $f(x)=y$ en temps $\leqslant P(|y|)$.
\item $\exists P$ polynome, $\forall y \in Im(f)$, il existe $x$, $|x| \leqslant P(|y|)$ tel que $f(x)=y$ car sinon on peut prendre $f(x)=\log |\log |x||$
\end{itemize}
\paragraph{Théorème :\\}
Il existe $f$ fonction à sens unique $\Longleftrightarrow P\neq NP$\\

$\Rightarrow$ : On construit le problème suivant :
\begin{itemize}
\item[I:] $x,y \in \Sigma^*,\ P$ polynome
\item[Q:] $\exists z \in \Sigma^*$, $f(xz)=y$ ? et $|z| \leqslant P(|y|)$
\end{itemize}
\begin{itemize}
\item problème dans NP ? Oui, z est le certificat
\item Si f à sens unique, le problème $\not\in P$. Sinon, on calcule rapidement un antécédent de $y$. On prend $y$ et on pose $Q$ avec $x=0$ si non $x=1$ si oui, $x=10$ ou $x=11$ ? Et ainsi de suite...\\
$\#$appels au problème de décision est de $2p(|y|)$.
\end{itemize}
$\Leftarrow$ Soit $\varphi : \Sigma^* \rightarrow \Sigma^* \times \Sigma^* $ une bijection, calculable en temps polynomial. 
$$ \{0,1\}^* \rightarrow \{0,1\}^* \times \{0,1\}^* $$
Par exemple : $ E \subset \{0,1\}^* \rightarrow \mathbb{N} $
$ \mbox{ écriture en base 2 } \leftarrow n $\\
$L \in NP$ avec $(M_L,p_L())$, $z\in \Sigma^* : z \longmapsto \varphi(z) = (\varphi_1(z), \varphi_2(z))=(x,y)$
$f(z)=\left\{ \begin{array}{l}
1) x \in L, y \mbox{ certificat de x alors } f(z)=x \\
2) \mbox{ sinon f n'est pas définie en z } \end{array} \right. $
\begin{enumerate}
\item $f$ calculable
\item $f$ à sens unique car si $\exists M$ tel que $x\mapsto z$ tel que $f(z)=x$ alors $x : M \mapsto z \mapsto \varphi(z) = (x,y)$ avec y certificat de $x\in L$. A partir de $x$, on trouverait le certificat, M fabriquerait donc des certificats ce qui est impossible.
\end{enumerate}
\paragraph{Variantes de SAT}
$ SAT \propto 3-SAT $ (et $ 3-SAT \propto SAT$)\\
Qu'en est -il de $2-SAT$ ? $\longrightarrow 2-SAT \in P$\\
Autre variante : SATC (Satisfaisabilité en circuit)

\begin{figure}[h!]
	\centering
  \scalebox{0.5}{\input{schema_SATC.pdf_t}}
\end{figure}

\paragraph{Proposition :\\}
$SAT \propto SATC$ \\
$(x_1 \cup x_2) \cap (x_1 \cup \bar{x_2} \cup x_3) \longrightarrow C $

\begin{figure}[h!]
	\centering
  \scalebox{0.5}{\input{schema_circuit.pdf_t}}
\end{figure}

\paragraph{Proposition :\\}
$SATC \propto SAT $
\paragraph{Réduction :\\}
Il faut ajouter des variables auxiliaires, autant que d'arêtes (arcs) issues des portes du circuit.
\subparagraph{Porte Non\\}
$y=\bar{x_1} \Leftrightarrow (x_1\cup y) \cap (\bar{x_1} \cup \bar{y}) = 1 $

\begin{figure}[h!]
	\centering
  \scalebox{0.7}{\input{negation.pdf_t}}
\end{figure}

\subparagraph{Porte Et}
$z=x\cap y \Leftrightarrow (\bar{x} \cup \bar{y} \cup z) \cap (x \cup \bar{y} \cup \bar{z}) \cap (\bar{x} \cup y \cup \bar{z}) \cap (x \cup y \cup \bar{z}) = 1 $
 
\begin{figure}[h!]
	\centering
  \scalebox{0.7}{\input{et.pdf_t}}
\end{figure}

\subparagraph{Porte Ou}
$z=x \cup y \Leftrightarrow (x\cup y \cup \bar{z}) \cap (x \cup \bar{y} \cup z) \cap(\bar{x} \cup y \cup z)\cap(\bar{x}\cup \bar{y} \cup z) =1 $

\begin{figure}[h]
	\centering
  \scalebox{0.7}{\input{ou.pdf_t}}
\end{figure}

\paragraph{Théorème de Cook :\\}
SAT est NP-complet.
\paragraph{Preuve :\\}
Il suffit de montrer que SATC est NP-complet.
\subparagraph{Stratégie :\\}
Le circuit a pour entrées $x_1 \ldots x_n y_1 \ldots y_m $ et le circuit renvoie 1 si $y_1 \ldots y_m$ est un certificat de $x$. Si $x \in L$, $ \exists y$ tel que $|y|\leqslant p_L(|x|)$, $M$ accepte $(x,y)$ en temps $\leqslant p_L(|x|)$.
\subparagraph{Tableau de calcul :\\}
Paramètre : $t=p_L(|x|)$
$$\begin{array}{|c|c|c|c|c|c|c|c|c|c|c|c|} \hline
-t & \ldots & \ldots & \ldots & -1 & 0& 1 & 2 & \ldots & n & \ldots  & t \\\hline
 \ldots & \ldots & \ldots & \ldots & y_1 & b & x_1 & x_2 & \ldots & x_n &  \ldots & \ldots \\\hline
\end{array}$$
On construit le tableau $(T_{ij})_{i=1\ldots t, j = -t \ldots t} $ de taille $t \times 2t+1$.

\begin{figure}[h]
	\centering
  \scalebox{0.5}{\input{tableau.pdf_t}}
\end{figure}

On a $T_{ij} =(a_{ij}, h_{ij}, q_{ij})$ avec :
\begin{itemize}
\item $a_{ij} =$ contenu de la case $j$ au temps $i$
\item $h_{ij} = \left\{ \begin{array}{l}
1 \mbox{ si le curseur est en j au temps i } \\
0 \mbox{ sinon} \end{array} \right. $
\item $q_{ij} =  \left\{ \begin{array}{l}
\mbox{état dans lequel est M au temps i si } h_{ij}=1 \\
0 \mbox{ sinon} \end{array} \right. $
\end{itemize}

\subparagraph{Point cl: \\}

\begin{figure}[h!]
	\centering
  \scalebox{0.7}{\input{Tij.pdf_t}}
\end{figure}

$T_{i+1\ j}$ est fonction déterministe de $T_{i\ j-1}, T_{ij}, T_{i\ j+1}$
\subparagraph{Circuit :\\}
entrée : $x_1, \ldots, x_n$ (constantes), $y_1, \ldots, y_m \in \Sigma \cup \{b\}$. \\
Toute fonction $F \rightarrow F$ est représentable par un circuit fini. A la fin, la sortie est 1 s'il existe $q_A$ dans la dernière ligne.

\paragraph{2-SAT :}
\subparagraph{Proposition :}
2-SAT est polynomial.
\begin{itemize}
\item[I:] $f=C_1 \cap C_2 \cap \ldots \cap C_k$ avec $C_i = x \cup y $
\item[Q:] f satisfaisable ?
\end{itemize}
\subparagraph{Exemple :} $(x \cup y) \cap (\bar{x} \cup y) \cap (x \cup \bar{y}) \cap (\bar{x} \cup \bar{y})$\\
Idée : graphe orienté avec pour sommets les litéraux. On a : $a \cup b$ devient $\bar{a} \rightarrow b$ et $ \bar{b} \rightarrow a$

\begin{figure}[h!]
	\centering
  \scalebox{0.7}{\input{graphe_oriente.pdf_t}}
\end{figure}


\subparagraph{Proposition :\\} 
f est satisfaisable si et seulement si $\nexists x$ tel que $x \rightarrow \bar{x}$ et $ \bar{x} \rightarrow x$.

\section{Pspace, RP, BPP}
\subsection{Pspace}
Au dessus de NP, on a Pspace, espace polynomial.
\paragraph{Définition :} Pspace \\
Pspace est la classe des langages L tels que $\exists M$ et un polynome P tels que sur l'entrée $(x_1,\ldots,x_n),\ \forall x \in \Sigma^*,\ M$ décide si $x \in L$ en temps $\leqslant 2^{q(|x|)}$ avec q un polynome. $|\Gamma|^{2m+1}|\mbox{états}|(2m+1)$ majore le nombre d'itérations (sinon la machine boucle).

\paragraph{Proposition :} $NP \subset Pspace $
\paragraph{Preuve :\\}
Soit $L \in NP$. L'algorithme est de passer en revue tous les certificats y possibles.

\paragraph{Conjecture :\\}
$ P \subsetneq NP \subsetneq Pspace $

\paragraph{Définition :\\}
Un langage L est dit Pspace-complet si :
\begin{itemize}
\item $L \in $ Pspace 
\item $\forall L' \in$ Pspace, $L' \propto L$
\end{itemize}
\paragraph{Théorème :\\}
FBQ est Pspace-complet.
\begin{itemize}
\item[I:] $\forall x_1,\ \exists x_2, x_3,\ \forall x_4,\ f(x_1,x_2,\ldots,x_n) = 1 $
\item[Q:] vrai ou faux ?
\end{itemize}

\subsection{Algorithmes probabilistes polynomiaux}
\paragraph{Exemple :} Test de (non)-primalité\\
Solovay - Strassen : \\
On a en entrée $n$, avec $a$ aléatoire, a-t-on $a^{\frac{n-1}{2}} = \left(\frac{a}{n}\right) \mod n$ ? Si non, $n$ n'est pas premier.
Si $n$ n'est pas premier, on a :
$$ \frac{\#\{a \in (\mathbb{Z}/n\mathbb{Z})^*,\ a^{\frac{n-1}{2}} \neq \left(\frac{a}{n}\right) \mod n \}}{\varphi(n)} \geqslant \frac{1}{2} $$

\paragraph{Définition :} RP \\
Soit $L$ un langage, $\exists M$ et $p$ un polynome tels que pour toute entrée $x$, il existe un ensemble de certificats $Y$, de taille $\leqslant p(|x|)$ tel que pour tout $(x,y)$ donnés à $M$ :
\begin{itemize}
\item si $x \not\in L$, $M$ rejette $(x,y)$ avec une probabilité $1$ (pour tout $y \in Y$)
\item si $x \in L$, $M$ accepte $(x,y)$ avec une probabilité $\geqslant \frac{1}{2} $ (pour une moitié des $y$).
\end{itemize}
M calcule en temps polynomial.

\paragraph{Proposition :\\}
$ RP \subset NP $ et $P \subset RP$

\paragraph{Définition :} BPP\\
Soit un langage $L$, $\exists M$ et un polynome $p$ tel que pour toute entrée $x$, il existe un ensemble de certificats $Y$, de taille $\leqslant p(|x|)$, tel que pour tout $(x,y)$:
\begin{itemize}
\item si $x \not\in L$, $M$ rejette $(x,y)$ avec une probabilité $> \frac{2}{3} $
\item si $x \in L$, $M$ accepte $(x,y)$ avec une probabilité $> \frac{2}{3} $
\end{itemize}

On a : $RP \subset BPP$ et $ RP \subset NP $
\end{document}